\documentclass[10pt,a4paper]{article}
\usepackage[utf8]{inputenc}
\usepackage[croatian]{babel}
\usepackage{amsmath}
\usepackage{amsfonts}
\usepackage{amsthm}
\usepackage{amssymb}
\usepackage{algorithm2e} %sudo apt-get install texlive-science
\usepackage{graphicx} 
\usepackage{hyperref}
\usepackage{listings} 
\usepackage{color}

%--------- CODE SETTINGS--------------------%
\definecolor{mygreen}{rgb}{0,0.6,0}
\definecolor{mygray}{rgb}{0.5,0.5,0.5}
\definecolor{mymauve}{rgb}{0.58,0,0.82}

\lstset{ %
  backgroundcolor=\color{white},   % choose the background color; you must add \usepackage{color} or \usepackage{xcolor}
  basicstyle=\footnotesize,        % the size of the fonts that are used for the code
  breakatwhitespace=false,         % sets if automatic breaks should only happen at whitespace
  breaklines=true,                 % sets automatic line breaking
  captionpos=b,                    % sets the caption-position to bottom
  commentstyle=\color{mygreen},    % comment style
  deletekeywords={...},            % if you want to delete keywords from the given language
  escapeinside={\%*}{*)},          % if you want to add LaTeX within your code
  extendedchars=true,              % lets you use non-ASCII characters; for 8-bits encodings only, does not work with UTF-8
  frame=single,                    % adds a frame around the code
  keepspaces=true,                 % keeps spaces in text, useful for keeping indentation of code (possibly needs columns=flexible)
  keywordstyle=\color{blue},       % keyword style
  language=C,                      % the language of the code
  morekeywords={*,...},            % if you want to add more keywords to the set
  numbers=left,                    % where to put the line-numbers; possible values are (none, left, right)
  numbersep=5pt,                   % how far the line-numbers are from the code
  numberstyle=\tiny\color{mygray}, % the style that is used for the line-numbers
  rulecolor=\color{black},         % if not set, the frame-color may be changed on line-breaks within not-black text (e.g. comments (green here))
  showspaces=false,                % show spaces everywhere adding particular underscores; it overrides 'showstringspaces'
  showstringspaces=false,          % underline spaces within strings only
  showtabs=false,                  % show tabs within strings adding particular underscores
  stepnumber=2,                    % the step between two line-numbers. If it's 1, each line will be numbered
  stringstyle=\color{mymauve},     % string literal style
  tabsize=2,                       % sets default tabsize to 2 spaces
  title=\lstname                   % show the filename of files included with \lstinputlisting; also try caption instead of title
}

%----------LOCAL SETTINGS---------------------%
\newtheorem{definicija}{Definicija}
\newtheorem{lema}{Lema}
\newtheorem{teorem}{Teorem}
\newtheorem{primjer}{Primjer}
%----------------------------------------------%

\title{Algoritmi}
%\author{}

\begin{document}
% ------------title page------------------------%
\maketitle
\tableofcontents
\newpage
%-----------------------------------------------%
% Author: Anto Čabraja
%-----------------------------------------------------------
\section{Algoritmi sa slučajnim faktorom i vjerojatnosna analiza} % (fold)
\label{sec:algoritmi_sa_slu_ajnim_faktorom_i_vjerojatnosna_analiza}

\begin{abstract}
U ovom poglavlju pokazat ćemo kako pretpostavke o distribuciji ulaznih podataka mogu pomoći u određivanju
prosječne složenosti algoritama. Također pokazat ćemo kako se može osigurati da se dobra svojstva distribucije 
održe te tako uvesti pojam očekivane složenosti. Cjelokupnu analizu ćemo provesti kroz jednostavan problem i 
pokazati nekoliko algoritama koji služe za generiranje permutacije niza elemenata. Na kraju ćemo se dotaknuti nekih zanimljivih tema iz ovog područja kao što su online modeli algoritama\\
\textbf{Ključne riječi:} algoritmi, pseudo-slučajni brojevi, vjerojatnosna analiza, karakteristična funkcija, očekivanje, prosječna složenost, očekivana složenost
,slučajni efekt 
\end{abstract}
\subsection{Uvod u problem}
Tvrtka X d.o.o. raspisala je natječaj za radno mjesto asistenta. Na tom položaju već imaju 
zaposlenu osobu no ta osoba je dobila unaprijeđenje te sada moraju pronaći adekvatnu osobu 
koja bi ga zamjenila. Pod adekvatna se podrazumjeva da ta nova osoba mora biti boja ili barem 
jednako dobra kao i dosadađnji asistent u obavljanju svih svojih dužnosti. Olakotna okolnost je 
ta što ljudi koji rade u ljudskim resursima  su jako dobri i iz intervjua mogu odmah zaključiti 
da je novi kandidat adekvatan ili ne.
Kako bi što prije pronašli zamjenu angažirali su agenciju za traženje radnika. 
Agencija svaki dan šalje jednog kandidata kojeg ljudi iz tvrtke ispitaju i ako je 
adekvatan odmah bude primljen na posao. Broj kandidta koje će agencija slati je unaprijed 
određen i za svakog novog kandidata tvrtka mora platiti uslugu agenciji. cijena te usluge je
mala ali nije zanemariva. S druge strane ako se tvrtka odluči na zapošljavanje novog kandidata 
to znači da prijašnjem radniku trebaju dati otkaz ili pronaći trugo radno mjesto što je skupo i novčano i vremenski.
Problem kojeg naša tvrtka želi rješiti je kako ispitati sve kandidata u svrhu pronalaska najadekvatnijeg 
ali tako da što manje puta moraju mjenjati trenutnog radnika.\\

\subsection{Implementacija i složenost}

\subsubsection{Implemetacija} 

Zapravo tvrtka X d.o.o. ima problem koji se jednostavno može zapisati sljedećim pseudokodom.
\begin{algorithm}[H]
\caption{Ispitaj sve kandidate}
najbolji = 0\;
\While{ agencija salje kandidata i }{
kvaliteta = IspitajKandidata(i)\;
\If{ kvaliteta veci od  najbolji}{
najbolji = kvaliteta\;
Zaposli(i)\;
}
}
\end{algorithm}

\subsubsection{Analiza složenosti}
Na početku ćemo predpostaviti da imamo $n$ kandidata. Dakle standardnom analizom složenost našeg 
problema bi bila $O(n)$. Imamo jednu petlju po svim kandidatima i pri tome biramo najboljeg. 
Problem je u tome što je cijena funkcije \textit{Zaposli()} velika i složenost našeg algoritma 
zapravo ovisi ponajviše o toj funkciji.\\
Označimo sa $t_i$ težinu koju ima funkcija \textit{Ispitajkandidata()}, a da $t_z$ težinu funkcije 
\textit{Zaposli()}. Sada složenost možemo pisati kao $O(t_in + t_z1m)$ gdje je $m$ zapravo koliko 
smo puta zadovoljili uvjet to jest zaposlili novog kandidata. Sada vidimo da je $t_in$ fiksan i ne 
može se izbjeći ali na $t_zm$ možemo pokušati utjecati.
\subsubsection*{Najgori slučaj}
Kako smo već vidjeli naša složenost će uvelike ovisiti o veličini $t_zm$. 
Tako da ćemo u daljnjem razmatranju složenost i izražavati preko tog koeficijenta. Dakle naš problem 
je složenosti $O(t_zm)=O(m)$. \\
U najgorem slučaju kada na ulaz u algoritam pristižu kandidati u takvom redosljedu da je svaki sljedeći 
kandidat bolji od prethodnog, tada je neizbježno svakog kandidata zaposliti i tako dobijemo $m=n$. 
To zapravo znači da je složenost $O(n)$.

\subsubsection*{Prosječne složenosti}
Da bi se ispitala prosječna složenost algoritma ovog tipa potrebno bi bilo izračunati složenost za sve 
moguće ulazne parametre i tek onda bi mogli govoriti o globalnoj prosječnoj složenosti. Ovdje ćemo se 
baviti vjerojatnosnom analizom kao metodom za traženje prosječne složenosti našeg algoritma. Vjerojatnosne 
metode se mogu koristiti na specificiranim objektima algoritma koji nam daju najveću težinu. Konkretno u našem 
slučaju to bi bila funkcija \textit{Zaposli()}. Dakle mi ćemo računati kolika je vjerojatnost da sljedeći kandidat 
bude zaposlen. Da bi uopće mogli razmatrati bilo što sa nekakvom vjerojatnošću moramo znati ponešto o distribuciji
ulaznih parametara. Važno je napomenuti da je to uprincipu jako teško i da se u većini slučajeva ne može unaprijed
znati kakvi su ulazni parametri. No možemo razmotriti problem i vidjeti gdje je razumno predpostaviti da je recimo
ulaz slučajno generiran, a da to ne utječe na svojstva algoritma. Konkretno u našem razmatranom problemu prirodno
je reći da kandidati stižu u nekom slučajnom poretku. Što to točno znači za naš problem?
Predpostavimo da svakom kandidatu odredimo ocjenu tj. da je svaki kandidat reprezentiran nekim brojem. Naš je posao 
zapravo naći maksimalni element tako da probemo po cjelom polju. Na ulazu nam se nalazi neka permutacija $n$ brojeva. 
To zapravo znači da ako kažemo da naši kandidati dolaze u slučajnom poretku to znači da je svaka permutacija od njih $n!$ 
jednako vjerojatna. U tom slučaju govorimo o uniformnoj slučajnoj permutaciji.

\subsection{Utjecaj slučajnih faktora u algoritmima}
Algoritmi sa slučajnim faktorm su zapravo algoritmi koji na neki način koriste slučajne brojeve. Neformalna i gruba definicija 
za randomizirane algoritme bi bila da su to algoritmi čije ponašanje nije određeno samo determinističkim poretkom ulaznih 
parametara nego utjecaj na ponašanje ima i slučajni faktor.Konkretno slučajni faktor je neki generator slučajnih brojeva. 
Generatora ima puno, a mi se bavimo onima koji će genrirati uniformne brojeve u nekom intervalu tj svakom broju daju jednaku 
vjerojatnost pojavljivanja. Takvi algoritmi trebaju permutirati ulazni niz podataka ili nekakav podniz podataka što također 
ima svoju složenost. Ponekad je ta složenost veća od složenosti samog algoritma.

\subsection{Složenost algoritma Ispitaj sve kandidate}
\subsubsection{Karakteristična funkcija slučajnih varijabli}
Karakteristična funkcija slučajne varijable je jednostavna metoda za definiranje odnosa između vjerojatnosti i očekivanja.
\begin{definicija}\label{definija_kfsv}
Karakterističnu funkciju slučajne varijable $I\{A\}$
definiramo kao :
\[ I\{A\} = \left\{ 
  \begin{array}{l l}
    1 & \quad \text{ako se A dogodio}\\
    0 & \quad \text{inače}
  \end{array} \right.\]
gdje je $S$ prostor događaja, a $A$ neki događaj u $S$.
\end{definicija}
Često umjesto $I\{A\}$ pišemo $X_A$.
\begin{lema}\label{lema_o_ocekivanju} 
Neka je $S$ prostor događaja, a $A$ neki događaj u $S$. Tada vrijedi $E[X_A]=P(X)$
\end{lema}
\begin{proof}
Po definiciji karakteristične funkcije slučajne varijable i definiciji očekivanja imamo sljedeće:
\begin{align*}
E[X_A] = E[I\{A\}] = 1\cdot P(A) + 0\cdot P(\overline{A}) = P(A) 
\end{align*}
Ovdje $\overline{A}$ označava komplement od $A$ u $S$
\end{proof}
Prava važnost karakteristične funkcije je kada trebamo izračunati očekivanje pojavljivanja nekog događaja ako neki postupak izvršavamo $n$ puta.
Kao primjer možemo uzeti bacanje novčića $n$ puta. I recimo da nas zanima koliki je očekivani broj pojavljivanja pisma. Sa $X_i$ označimo karakterističnu funkciju da je
u itom bacanju palo pismo, a sa $X$ označimo slučajnu varijablu koja predstavlja ukupan broj pisama u $n$ bacanja. očito viijedi sljedeće:
\begin{align*}
X = \sum_{i=1}^{n}X_i
\end{align*}
Ako želimo izračunati očekivanje od $X$ to nam zapravo znači
\begin{align*}
E[X] = E[\sum_{i = 1}^{n}X_i]
\end{align*}
Iz leme~\ref{lema_o_ocekivanju} i definicije računanja očekivanja lako dobijemo da je 
\begin{align*}
E[X] = E[\sum_{i = 1}^{n}X_i] = \sum_{i = 1}^{n}E[X_i] = \sum_{i = 1}^{n}\frac{1}{2} = \frac{n}{2}
\end{align*}
Vidimo da se račun poklapa sa očekivanjem neovisnog bacanja novčića $n$ puta. Ovo bacanje novčića je slučajan događaj.

\subsubsection{Složenost algoritma za problem zapošljavanja}
Koristeći prethodno pokazani račun pokušat ćemo izračunati očekivani broj osoba koje će trebati zaposliti. To možemo jer smo 
prije rekli da je logično zaključiti da osobe dolaze u nekom slučajnom poretku i da je vjerojatnost prihvaćanja za svaku osobu ista.
Označimo sa $X$ slučajnu varijablu koja označava koliko puta smo zaposlili novog kandidata. Naravno sada možemo izračunati očekivanje od $X$
\begin{align*}
E[X] = \sum_{x=1}^{n}xP(X=x)
\end{align*}
No ovakav račun često zna biti težak za računanje pa ćemo stoga iskoriti alat koji smo definirali u prethodnoj cjelini.\\
Označimo sa $X_i$ događaj da li je prihvaćen $i-ti$ kandidat.
\begin{align*}
X_i = I\{\text{i-ti kandidat je prihvaćen}\}\\
X_i = \left\{ 
  \begin{array}{l l}
    1 & \quad \text{i-ti kandidat prihvaćen}\\
    0 & \quad \text{inače}
  \end{array} \right.
\end{align*}
Iz ovoga imamo da je $X = X_1 + X_2 + \cdots + X_n$\\
Iz leme~\ref{lema_o_ocekivanju} znamo da vrijedi $E[X_i] = P\{\text{kandidat i je prihvaćen}\}$.
Vjerojatnost da je $i-ti$ kandidat prihvaćen je situacija kada je on bolji od njih $i-1$, a kako svi imaju istu vjerojatnost prihavčanja slijedi da je
\\$P\{\text{kandidat i je prihvaćen}\}=1/i$. To znači da je $E[X_i] = 1/i$. Sada možemo izračunati $E[X]$.
\begin{align*}
E[X] = E[\sum_{i = 1}^{n}X_i] = \sum_{i = 1}^{n}E[X_i] = \sum_{i = 1}^{n}1/i = \ln(n) + O(1)
\end{align*}
Zadnji korak slijedi iz ocjene sume integralom.
Prethodnim računom smo dokazali sljedeću korisnu lemu.
\begin{lema}\label{slozenost_rand_alg}
Neka imamo n podataka i neka su oni u slučajnom poretku tada je prosječna složenost algoritma Ispitaj sve kandidate O($\ln(t_zn)$)  
\end{lema}
Konkretno ovim smo pokazali da je $m = \ln(n)$\\
Naravno ova lema se može poopćiti i na sve ostale slučajeve u kojima imamo situaciju da ulazni podatak prihvaćamo uz neke uvjete. 
Još jednom napominjemo da dobro treba proučiti ulazne podatke i vidjeti da li imaju neku poželjnu distribuciju koja će nam omogućiti ovakvu analizu prosječne složenosti.
\subsection{Algoritmi sa utjecajem slučajnih faktora}
U pethodnoj cjelini smo morali znati ponešto o distribuciji ulaznih podataka. Ovdje se pitamo da li je moguće permutirati ulazne podatke tako da 
dobijemo željenu distribuciju. Dakle zamislimo da u našem algoritmu Izaberi najboljeg kandidata znamo listu unaprijed i mi biramo kojeg ćemo kandidata pozvati sljedećeg.
Ako to predstavimo u terminima pseudo-algoritma onda naš algoritam izgleda ovako:
\begin{center}
\begin{algorithm}[H]
\caption{Randomizirani ispitaj sve kandidate}
Permuturaj ulazni vektor\;
najbolji = 0\;
\While{ uzmi sljedeceg po redu}{
kvaliteta = IspitajKandidata(i)\;
\If{ kvaliteta veca od  najbolji}{
najbolji = kvaliteta\;
Zaposli(i)\;
}
}
\end{algorithm}
\end{center}
Za ovaj algoritam također vrijedi da je složenost $O(t_zn)$ gdje je $n$ broj kandidata.\\
U čemu je onda zapravo razlika? Razlika je u tome što smo u prethodnoj cjelini govorili o prosječnoj složenosti, a ovdje želimo dobiti očekivanu složenost.
Prosječna složenost nam je dala mogućnost da algoritam poboljšamo, ona nam kaže da postoji mogućnost poboljšanja ali ne i kako. Ako gledamo da su sve permutacije 
na ulazu jednako vjerojatne onda je jednako vjerojatno da se naš algoritam izvši brzo ili sporo. No ako mi utječemo na ulaze podatke da budu u što uniformnijoj slučajnoj
distribuciji onda možemo očekivati da će se naš problem ponašati kao u prosječnoj složenosto. Željenu permutaciju možemo ostvariti nekim od algoritama za generiranje slučajnih nizova. To su algoritmi koji ulazni niz podataka permutiraju u neki slučajni poredak.
U ovom radu ćemo obraditi dva takva algoritma.\\
Prije nego uvedenom algoritme potrebno je objasniti dvije funkcije koje ćemo koristit. Radi se o funkciji RANDOM(a,b) i SORT(A,B). RANDOM() vraća jedan broj između a i b i to tako da je vjerojatnost vraćanja bilo kojeg broja jednaka(uniformna distribucija). Složenost ove funkcije je $O(1)$. SORT(A,B) sortira niz A tako da koristi elemente iz niza B kao vrijednosti po kojima sortira A. Složenost ove funkcije je ovisna o vrsti sortiranja što ćemo upoznati u narednim izlaganjima, ali pretpostavimo najgore a to je $O(n^2)$.Sada možemo napisati i analizirati algoritme algoritme


\begin{algorithm}[H]
\caption{Permutiraj po sortiranom}
n = A.size(); // A ulazni niz\;
P[] = new array()\;
\For {i = 1:n}{
  P[i]=RANDOM(1,$n^3$)
}
SORT(A,P);
\end{algorithm}

Ovaj algoritam radi tako da kreiramo polje slučajnih brojeva i da onda ulazni niz elemenata sortiramo koristeći to slučajno generirano polje. Ovdje je složenost velika
s obzirom da moramo raditi sortiranje koje je prosjećne složenosti $O(n\lg(n))$.
Ostaje samo provjeriti da li je ovaj algoritam korektan za naše potrebe. A to znači da li generira uniformnu permutaciju brojeva.
\begin{lema}
Agoritam permutiraj po sortiranom vraća uniformnu permutaciju ulaznih podataka
\end{lema}
\begin{proof}
Dokaz se može pogledati u knjizi Introduction to algorithms 3th edition, str.125-126
\end{proof}

\begin{algorithm}[H]
\caption{Automatsko pridruzivanje}
n = A.size(); // A ulazni niz\;
\For {i = 1:n}{
  A[i]=A[RANDOM(i,$n$)]
}
\end{algorithm}
Ovaj algoritam je puno brži nego prethodni i ima linearnu složenost. Radi tako da elementu $i$ pridružuje neki element iz ostatka polja. On ne generira sve moguće permutacije ali računa uniformnu slučajnu permutaciju.
\begin{lema}
Algoritam Automatsko pridruživanje računa uniformnu slučajnu permutaciju.
\end{lema}
\begin{proof}
Dokaz se može pogledati u knjizi Introduction to algorithms 3th edition, str.127
\end{proof}
% End author: Anto Čabraja
\subsection{Zanimljivosti}

\end{document}
\documentclass[10pt,a4paper]{article}
\usepackage[utf8]{inputenc}
\usepackage[croatian]{babel}
\usepackage{amsmath}
\usepackage{amsfonts}
\usepackage{amssymb}
\usepackage{algorithm2e} %sudo apt-get install texlive-science
\usepackage{graphicx} 
\usepackage{hyperref}
\usepackage{listings} 
\usepackage{color}

%--------- CODE SETTINGS--------------------%
\definecolor{mygreen}{rgb}{0,0.6,0}
\definecolor{mygray}{rgb}{0.5,0.5,0.5}
\definecolor{mymauve}{rgb}{0.58,0,0.82}

\lstset{ %
  backgroundcolor=\color{white},   % choose the background color; you must add \usepackage{color} or \usepackage{xcolor}
  basicstyle=\footnotesize,        % the size of the fonts that are used for the code
  breakatwhitespace=false,         % sets if automatic breaks should only happen at whitespace
  breaklines=true,                 % sets automatic line breaking
  captionpos=b,                    % sets the caption-position to bottom
  commentstyle=\color{mygreen},    % comment style
  deletekeywords={...},            % if you want to delete keywords from the given language
  escapeinside={\%*}{*)},          % if you want to add LaTeX within your code
  extendedchars=true,              % lets you use non-ASCII characters; for 8-bits encodings only, does not work with UTF-8
  frame=single,                    % adds a frame around the code
  keepspaces=true,                 % keeps spaces in text, useful for keeping indentation of code (possibly needs columns=flexible)
  keywordstyle=\color{blue},       % keyword style
  language=C,                      % the language of the code
  morekeywords={*,...},            % if you want to add more keywords to the set
  numbers=left,                    % where to put the line-numbers; possible values are (none, left, right)
  numbersep=5pt,                   % how far the line-numbers are from the code
  numberstyle=\tiny\color{mygray}, % the style that is used for the line-numbers
  rulecolor=\color{black},         % if not set, the frame-color may be changed on line-breaks within not-black text (e.g. comments (green here))
  showspaces=false,                % show spaces everywhere adding particular underscores; it overrides 'showstringspaces'
  showstringspaces=false,          % underline spaces within strings only
  showtabs=false,                  % show tabs within strings adding particular underscores
  stepnumber=2,                    % the step between two line-numbers. If it's 1, each line will be numbered
  stringstyle=\color{mymauve},     % string literal style
  tabsize=2,                       % sets default tabsize to 2 spaces
  title=\lstname                   % show the filename of files included with \lstinputlisting; also try caption instead of title
}

%----------LOCAL SETTINGS---------------------%
\newtheorem{definicija}{Definicija}
\newtheorem{lema}{Lema}
\newtheorem{teorem}{Teorem}
\newtheorem{primjer}{Primjer}
%----------------------------------------------%

\title{Algoritmi}
%\author{}

\begin{document}
% ------------title page------------------------%
\maketitle
\newpage
%-----------------------------------------------%
% Author: Anto Čabraja
%-----------------------------------------------------------
\section{Algoritmi sa slučajnim faktorom i vjerojatnosna analiza} % (fold)
\label{sec:algoritmi_sa_slu_ajnim_faktorom_i_vjerojatnosna_analiza}

\begin{abstract}
U ovom poglavlju promatramo svojstva algoritama u kojima se koriste 
generatori slučajnih brojeva kao alat za generiranje poretka ulaznih podataka. 
Pkazat ćemo nekoliko načina generiranja pseudo-slučajnih brojeva, te rješiti jedan 
konkretan problem gdje se ovaj tip algoritama koristi.
\end{abstract}
\subsection{Uvod u problem}
Tvrtka X d.o.o. raspisala je natječaj za radno mjesto asistenta. Na tom položaju već imaju 
zaposlenu osobu no ta osoba je dobila unaprijeđenje te sada moraju pronaći adekvatnu osobu 
koja bi ga zamjenila. Pod adekvatna se podrazumjeva da ta nova osoba mora biti boja ili barem 
jednako dobra kao i dosadađnji asistent u obavljanju svih svojih dužnosti. Olakotna okolnost je 
ta što ljudi koji rade u ljudskim resursima  su jako dobri i iz intervjua mogu odmah zaključiti 
da je novi kandidat adekvatan ili ne.
Kako bi što prije pronašli zamjenu angažirali su agenciju za traženje radnika. 
Agencija svaki dan šalje jednog kandidata kojeg ljudi iz tvrtke ispitaju i ako je 
adekvatan odmah bude primljen na posao. Broj kandidta koje će agencija slati je unaprijed 
određen i za svakog novog kandidata tvrtka mora platiti uslugu agenciji. cijena te usluge je
mala ali nije zanemariva. S druge strane ako se tvrtka odluči na zapošljavanje novog kandidata 
to znači da prijašnjem radniku trebaju dati otkaz ili pronaći trugo radno mjesto što je skupo i novčano i vremenski.
Problem kojeg naša tvrtka želi rješiti je kako ispitati sve kandidata u svrhu pronalaska najadekvatnijeg 
ali tako da što manje puta moraju mjenjati trenutnog radnika.\\

\subsection{Implementacija i složenost} 
Zapravo tvrtka X d.o.o. ima problem koji se jednostavno može zapisati sljedećim pseudokodom.

\begin{algorithm}[H]
\caption{Ispitaj sve kandidate}
najbolji = 0\;
\While{ agencija salje kandidata i }{
kvaliteta = IspitajKandidata(i)\;
\If{ kvaliteta veci od  najbolji}{
najbolji = kvaliteta\;
Zaposli(i)\;
}
}
\end{algorithm}

\subsection*{Analiza složenosti:}
Na početku ćemo predpostaviti da imamo $n$ kandidata. Dakle standardnom analizom složenost našeg 
problema bi bila $O(n)$. Imamo jednu petlju po svim kandidatima i pri tome biramo najboljeg. 
Problem je u tome što je cijena funkcije \textit{Zaposli()} velika i složenost našeg algoritma 
zapravo ovisi ponajviše o toj funkciji.\\
Označimo sa $t_i$ težinu koju ima funkcija \textit{Ispitajkandidata()}, a da $t_z$ težinu funkcije 
\textit{Zaposli()}. Sada složenost možemo pisati kao $O(t_in + t_z1m)$ gdje je $m$ zapravo koliko 
smo puta zadovoljili uvjet to jest zaposlili novog kandidata. Sada vidimo da je $t_in$ fiksan i ne 
može se izbjeći ali na $t_zm$ možemo pokušati utjecati.
\subsubsection*{Najgori slučaj}
Kako smo već vidjeli naša složenost će uvelike ovisiti o veličini $t_zm$. 
Tako da ćemo u daljnjem razmatranju složenost i izražavati preko tog koeficijenta. Dakle naš problem 
je složenosti $O(t_zm)=O(m)$. \\
U najgorem slučaju kada na ulaz u algoritam pristižu kandidati u takvom redosljedu da je svaki sljedeći 
kandidat bolji od prethodnog, tada je neizbježno svakog kandidata zaposliti i tako dobijemo $m=n$. 
To zapravo znači da je složenost $O(n)$.

\subsubsection*{Analiza prosječne složenosti}
Da bi se ispitala prosječna složenost algoritma ovog tipa potrebno bi bilo izračunati složenost za sve 
moguće ulazne parametre i tek onda bi mogli govoriti o globalnoj prosječnoj složenosti. Ovdje ćemo se 
baviti vjerojatnosnom analizom kao metodom za traženje prosječne složenosti našeg algoritma. Vjerojatnosne 
metode se mogu koristiti na specificiranim objektima algoritma koji nam daju najveću težinu. Konkretno u našem 
slučaju to bi bila funkcija \textit{Zaposli()}. Dakle mi ćemo računati kolika je vjerojatnost da sljedeći kandidat 
bude zaposlen. Da bi uopće mogli razmatrati bilo što sa nekakvom vjerojatnošću moramo znati ponešto o distribuciji
ulaznih parametara. Važno je napomenuti da je to uprincipu jako teško i da se u većini slučajeva ne može unaprijed
znati kakvi su ulazni parametri. No možemo razmotriti problem i vidjeti gdje je razumno predpostaviti da je recimo
ulaz slučajno generiran, a da to ne utječe na svojstva algoritma. Konkretno u našem razmatranom problemu prirodno
je reći da kandidati stižu u nekom slučajnom poretku. Što to točno znači za naš problem?
Predpostavimo da svakom kandidatu odredimo ocjenu tj. da je svaki kandidat reprezentiran nekim brojem. Naš je posao 
zapravo naći maksimalni element tako da probemo po cjelom polju. Na ulazu nam se nalazi neka permutacija $n$ brojeva. 
To zapravo znači da ako kažemo da naši kandidati dolaze u slučajnom poretku to znači da je svaka permutacija od njih $n!$ 
jednako vjerojatna. U tom slučaju govorimo o uniformnoj slučajnoj permutaciji.
\subsection{Agoritam sa slučajni brojevima\\ i njegova analiza}
Algoritmi sa slučajnim faktorm su zapravo algoritmi koji na neki način koriste slučajne brojeve. Neformalna i gruba definicija 
za randomizirane algoritme bi bila da su to algoritmi čije ponašanje nije određeno samo determinističkim poretkom ulaznih 
parametara nego utjecaj na ponašanje ima i slučajni faktor.Konkretno slučajni faktor je neki generator slučajnih brojeva. 
Generatora ima puno, a mi se bavimo onima koji će genrirati uniformne brojeve u nekom intervalu tj svakom broju daju jednaku 
vjerojatnost pojavljivanja. Takvi algoritmi trebaju permutirati ulazni niz podataka ili nekakav podniz podataka što također 
ima svoju složenost. Ponekad je ta složenost veća od složenosti samog algoritma.

\subsection{Složenost algoritma sa slučajnim faktorom}

\subsection{Generatori slučajnih brojeva}

% End author: Anto Čabraja


\end{document}